% Options for packages loaded elsewhere
\PassOptionsToPackage{unicode}{hyperref}
\PassOptionsToPackage{hyphens}{url}
%
\documentclass[
]{book}
\usepackage{amsmath,amssymb}
\usepackage{lmodern}
\usepackage{ifxetex,ifluatex}
\ifnum 0\ifxetex 1\fi\ifluatex 1\fi=0 % if pdftex
  \usepackage[T1]{fontenc}
  \usepackage[utf8]{inputenc}
  \usepackage{textcomp} % provide euro and other symbols
\else % if luatex or xetex
  \usepackage{unicode-math}
  \defaultfontfeatures{Scale=MatchLowercase}
  \defaultfontfeatures[\rmfamily]{Ligatures=TeX,Scale=1}
\fi
% Use upquote if available, for straight quotes in verbatim environments
\IfFileExists{upquote.sty}{\usepackage{upquote}}{}
\IfFileExists{microtype.sty}{% use microtype if available
  \usepackage[]{microtype}
  \UseMicrotypeSet[protrusion]{basicmath} % disable protrusion for tt fonts
}{}
\makeatletter
\@ifundefined{KOMAClassName}{% if non-KOMA class
  \IfFileExists{parskip.sty}{%
    \usepackage{parskip}
  }{% else
    \setlength{\parindent}{0pt}
    \setlength{\parskip}{6pt plus 2pt minus 1pt}}
}{% if KOMA class
  \KOMAoptions{parskip=half}}
\makeatother
\usepackage{xcolor}
\IfFileExists{xurl.sty}{\usepackage{xurl}}{} % add URL line breaks if available
\IfFileExists{bookmark.sty}{\usepackage{bookmark}}{\usepackage{hyperref}}
\hypersetup{
  pdftitle={10 Cool Things about R},
  pdfauthor={Yihui Xie},
  hidelinks,
  pdfcreator={LaTeX via pandoc}}
\urlstyle{same} % disable monospaced font for URLs
\usepackage{graphicx}
\makeatletter
\def\maxwidth{\ifdim\Gin@nat@width>\linewidth\linewidth\else\Gin@nat@width\fi}
\def\maxheight{\ifdim\Gin@nat@height>\textheight\textheight\else\Gin@nat@height\fi}
\makeatother
% Scale images if necessary, so that they will not overflow the page
% margins by default, and it is still possible to overwrite the defaults
% using explicit options in \includegraphics[width, height, ...]{}
\setkeys{Gin}{width=\maxwidth,height=\maxheight,keepaspectratio}
% Set default figure placement to htbp
\makeatletter
\def\fps@figure{htbp}
\makeatother
\setlength{\emergencystretch}{3em} % prevent overfull lines
\providecommand{\tightlist}{%
  \setlength{\itemsep}{0pt}\setlength{\parskip}{0pt}}
\setcounter{secnumdepth}{-\maxdimen} % remove section numbering
\ifluatex
  \usepackage{selnolig}  % disable illegal ligatures
\fi

\title{10 Cool Things about R}
\author{Yihui Xie}
\date{2021-07-17}

\begin{document}
\frontmatter
\maketitle

\mainmatter
\hypertarget{introduction}{%
\chapter{Introduction}\label{introduction}}

\hypertarget{who-is-this-for}{%
\section{Who is this for?}\label{who-is-this-for}}

You are doing research / data analysis with other tools but you are
curious about R. Maybe it's because\ldots{}

\begin{itemize}
\tightlist
\item
  FOMO - You want to know what all the fuzz is about?
\item
  You feel that your current workflow isn't efficient, robust, or
  reproducible enough?
\item
  There's this \emph{one} R package you would really like to use\ldots?
\item
  You tried R before and were overwhelmed, but would like to give it a
  second chance?
\end{itemize}

Or, you are using R already, but you feel you lost track of all the
recent changes and want an update on the state-of-the-Rt?

\hypertarget{im-assuming}{%
\subsection{I'm assuming \ldots{}}\label{im-assuming}}

\begin{itemize}
\tightlist
\item[$\boxtimes$]
  You have no real programming experience
\item[$\boxtimes$]
  You are not particularly interested in computers or programming
\item[$\boxtimes$]
  You want to get things done and you believe that R might help you with
  that
\end{itemize}

\hypertarget{who-am-i}{%
\section{Who am I?}\label{who-am-i}}

I have worked with R and other languages (Matlab, Python, \ldots)
professionally in a research context for almost 10 years. I sometimes
teach R with
\href{https://www.ub.uio.no/english/writing-publishing/dsc/carpentry-uio/}{The
Carpentries} (usually half/full day courses). I have also written two R
packages, \href{https://teebusch.github.io/mifa/}{\texttt{\{mifa\}}} and
\href{https://teebusch.github.io/noah/}{\texttt{\{noah\}}}, and
contributed to a few others (e.g,
\href{https://github.com/patrickreidy/textgRid}{\texttt{\{TextGrid\}}},
\href{https://github.com/HomeBankCode/rlena}{\texttt{\{rlena\}}})

\hypertarget{what-is-r}{%
\section{What is R?}\label{what-is-r}}

R is a statistical computing and data analysis environment that is
\href{https://www.tiobe.com/tiobe-index/r/}{widely used} in academia
(bioinformatics, geoscience, digital humanities, \ldots) and industry
(finance, biosciences, pharmacology, \ldots). It is a free and open
source implementation of the
\href{https://en.wikipedia.org/wiki/S_(programming_language)}{\emph{S}
language}, which was created in the 1970s with \emph{interactive data
analysis} in mind. That means:

\begin{itemize}
\tightlist
\item
  \textbf{Interactive} -- R works like a really fancy pocket calculator
  that lets you make calculations, explore and plot data quickly and
  ``on the fly.''
\item
  \textbf{Focussed on Data Analysis} -- More than in many other
  programming languages, data and statistics are at the core of R. That
  is, there are built-in data structures and functions that are meant to
  make data analysis easy.
\end{itemize}

Some say R is a
\href{https://en.wikipedia.org/wiki/Domain-specific_language}{domain
specific programming language}. Others say it's
\href{https://youtu.be/6S9r_YbqHy8}{an environment for interactive data
analysis that \emph{has} a programming language}. In any case, R can be
a versatile \& powerful \emph{one-stop-shop} for the whole data analysis
workflow:

\begin{figure}
\centering
\includegraphics{https://i.imgur.com/sclqeRc.jpg}
\caption{The data science workflow}
\end{figure}

\begin{itemize}
\tightlist
\item
  \textbf{Data collection}
\item
  \textbf{Data cleaning \& transformation} -- also known as data
  wrangling or data munging
\item
  \textbf{Data Visualisation}
\item
  \textbf{Modeling} -- using classical statistics as well as machine
  learning
\item
  \textbf{Communication} -- e.g., in reports, papers, presentations or
  dashboards
\end{itemize}

Data analysis is certainly R's core application. However, some people
claim that \href{https://youtu.be/m6nUdoP6894}{you can use R for
everything\ldots{}}

\begin{center}\rule{0.5\linewidth}{0.5pt}\end{center}

\hypertarget{cool-things-about-r}{%
\section{10 cool things about R}\label{cool-things-about-r}}

There are a lot of good reasons to learn R! Here are the 10 things that
I like most about R:

\begin{enumerate}
\def\labelenumi{\arabic{enumi}.}
\tightlist
\item
  Free and open source
\item
  Easy to get started
\item
  Flexible language
\item
  Well documented
\item
  Expandible
\item
  Powerful data wrangling
\item
  Powerful data visualisation
\item
  Powerful statistical methods
\item
  Encourages reproducible research
\item
  Awesome community
\end{enumerate}

We will have a closer look at all of these, but before we start, let's
get something out of the way: There are some very good reasons
\emph{not} to learn R!

\hypertarget{methods}{%
\chapter{Methods}\label{methods}}

We describe our methods in this chapter.

\backmatter
\end{document}
